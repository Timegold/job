%% The class has an option which changes the formatting of the chapter
%%% opener and part page. Follow the \documentclass command with
%%% [opener-a]
%%% [opener-b]
%%% [opener-c]
%%% [opener-d]
%%% to switch between the background graphics. Leave the option blank for
%%% a default, "graphic-less" part and chapter opener.

% thesis version

%\documentclass[final,11pt,onecolumn,a4paper,twoside]{scrbook}
\documentclass[final,11pt,onecolumn,a4paper,twoside]{scrbook_gj}
%\documentclass[draft,10pt,onecolumn,a4paper,twoside]{scrbook_gj}
%{newsiambook}

\usepackage{chngcntr}
\counterwithout{section}{chapter}
\counterwithout{figure}{chapter}
\setcounter{secnumdepth}{3}
%\setcounter{secnumdepth}{2}
\setcounter{tocdepth}{3}
\usepackage{amsmath}


\iffalse
 \newtheorem{theorem}{Theorem}[chapter]
  \newtheorem{lemma}{Lemma}[chapter]
  \newtheorem{corollary}{Corollary}[chapter]
  \newtheorem{proposition}{Proposition}[chapter]
  \newtheorem{definition}{Definition}[chapter]
\fi

\usepackage{afterpage}
\usepackage[]{setspace}
\onehalfspacing
\usepackage[sort&compress,sectionbib]{natbib}
\bibpunct{(}{)}{;}{a}{}{,}
%\usepackage[LY1]{fontenc}
%\usepackage[mtbold,LY1]{siammathtime}
%\usepackage{bm}
%%%\usepackage{epsfig}
%%%\usepackage{graphicx}
%%%\usepackage{graphics}
%%%\usepackage{float}
%%%\usepackage{floatfig}
%\restylefloat{figure}
\usepackage{makeidx}
\usepackage{multicol}
%\usepackage{subeqn}
%\usepackage{chapterbib}
%\usepackage{lucidbrb}
\usepackage{palatino}
\usepackage{amssymb}
%\usepackage{amsfont}
\usepackage{amsmath}
%\usepackage{array}
%\usepackage{appendix}
\usepackage{rotating}
%%%\usepackage{layout}



% general newcommands
\newcommand{\kmskpc}{km\, s$^{-1}$\, kpc$^{-1}$}
\newcommand{\av}{$A_V$}
\newcommand{\los}{line-of-sight}
\newcommand{\vtheta}{$v_{\theta}$}
\newcommand{\vrad}{$v_{\mathrm{r}}$}
\newcommand{\vlos}{$v_{\mathrm{los}}$}
\newcommand{\vc}{$v_{\mathrm{circ}}$}
\newcommand{\Msun}{$\mathcal{M}_\odot$}
\newcommand{\Rsun}{$R_{\odot}$}
\newcommand{\Rbar}{$R_{\mathrm{bar}}$}

\def\Lsun{\hbox{\it L$_\odot$}}
\def\Mbol{\hbox{\it M$_{\rm bol}$}}
\def\Msun{\hbox{\it M$_\odot$}}
\def\Teff{\hbox{\it T$_{\rm eff}$}}



\newcommand{\um}{$\mu$m}
\newcommand{\kms}{km\thinspace s$^{-1}$}
\newcommand{\htoo}{H$_{\rm 2}$O}
\newcommand{\hcn}{H$^{\rm 13}$CN}


\newcommand{\be}{\begin{equation}}
\newcommand{\ee}{\end{equation}}
\newcommand{\ba}{\begin{eqnarray}}
\newcommand{\ea}{\end{eqnarray}}
\newcommand{\etal}{et al.}
\newcommand{\Ex}{\times10^}
\newcommand{\comma}{\,,}
\newcommand{\fullstop}{\,.}
\newcommand{\onu}{_{\nu}}
\newcommand{\jbar}{\overline{J}}
\renewcommand{\textfraction}{0}

%\setlength{\oddsidemargin}{54pt}
%\setlength{\evensidemargin}{55pt}
%\setlength{\paperwidth}{505pt}
%\setlength{\paperheight}{722pt}
%\setlength{\textwidth}{355pt}
%\setlength{\textheight}{538pt}

% proefschrift formaat

\setlength{\oddsidemargin}{54pt}
\setlength{\evensidemargin}{55pt}
\setlength{\textwidth}{15.cm}
\setlength{\textheight}{21.0cm}
\setlength{\paperwidth}{18.0cm}
\setlength{\paperheight}{24.0cm}

% refereed version

%\setlength{\oddsidemargin}{5pt}
%\setlength{\evensidemargin}{5pt}
%\setlength{\textwidth}{15cm}
%\setlength{\textheight}{22.0cm}
%\setlength{\paperwidth}{18.0cm}
%\setlength{\paperheight}{26.0cm}

%\usepackage{crop} 
%\crop
\makeindex
% definitions used by included articles, reproduced here for 
% educational benefit, and to minimize alterations needed to be made
% in developing this sample file.

\newcommand{\pe}{\psi}
\def\d{\delta} 
\def\ds{\displaystyle} 
\def\e{{\epsilon}} 
\def\eb{\bar{\eta}}  
\def\enorm#1{\|#1\|_2} 
\def\Fp{F^\prime}  
\def\fishpack{{FISHPACK}} 
\def\fortran{{FORTRAN}} 
\def\gmres{{GMRES}} 
\def\gmresm{{\rm GMRES($m$)}} 
\def\Kc{{\cal K}} 
\def\norm#1{\|#1\|} 
\def\wb{{\bar w}} 
\def\zb{{\bar z}} 
% some definitions of bold math italics to make typing easier.
% They are used in the corollary.

\def\bfE{\mbox{\boldmath$E$}}
\def\bfG{\mbox{\boldmath$G$}}


% making the postscript output mode 

\def\exeps{eps}
\def\exps{ps}

% making the pdf output mode 


%\def\exeps{pdf}
%\def\exps{pdf}

%\rmfamily
%Serif Font Family
%\sffamily
%Sans Serif Font Family
%\ttfamily
%Monospaced Font Family
\usepackage{helvet}
\renewcommand{\familydefault}{\sfdefault}


\def\mnheh{\hbox{n$_{\rm He}$/n$_{\rm H}$}}
\def\nnihe{\hbox{n$_{\rm N}$/n$_{\rm He}$}}
\def\Mdot{\hbox{$\dot {M}$}}
\def\Zdot{\hbox{$\dot {Z}$}}
\def\Rsun{\hbox{\it R$_\odot$}}
\def\Zsun{\hbox{\it Z$_\odot$}}
\def\Rstar{\hbox{\it R$_*$}}
\def\Lsun{\hbox{\it L$_\odot$}}
\def\Lstar{\hbox{\it L$_*$}}
\def\Lx{\hbox{\it L$_X$}}
\def\Lbol{\hbox{\it L$_{bol}$}}
\def\Mbol{\hbox{\it M$_{bol}$}}
\def\Msun{\hbox{\it M$_\odot$}}
\def\Minit{\hbox{\it M$_{\rm initial}$}}
\def\Msunyr{\hbox{\it M$_\odot\,$yr$^{-1}$}}
\def\Myr{\hbox{\it Myr}}
\def\Gyr{\hbox{\it Gyr}}
\def\kpc{\hbox{\it kpc}}
\def\Teff{\hbox{\it T$_{\rm eff}$}}
\def\Vinf{\hbox{$v_\infty$}}
\def\I{\hbox{\it I}}
\def\J{\hbox{\it J}}
\def\H{\hbox{\it H}}
\def\K{\hbox{\it K}}
\def\mk{\hbox{\it K}}
\def\Mk{\hbox{\it M$_{\rm K}$}}
\newcommand{\Ks}{{\it K$_{\rm s}$}}
\newcommand{\Al}{{\it A$_\lambda$}}
\newcommand{\Aks}{{\it A$_{\rm K_{\rm s}}$}}
\newcommand{\Ak}{{\it A$_{\rm K}$}}
\newcommand{\Av}{{\it A$_{\rm V}$}}
\newcommand{\sig}{$\sigma_{\rm A_{\rm K_{\rm s}}}$}
\def\BCK{\hbox{\it BC$_{\rm K}$}}
\def\BCV{\hbox{\it BC$_{\rm V}$}}
\def\simgr{\mathrel{\hbox{\rlap{\hbox{\lower4pt\hbox{$\sim$}}}\hbox{$>$}}}}
\def\HH{H{\sc ii}}	% HII region
\def\ew{\hbox{\it EW$_{\rm 1.87~\micron}$}}
\def\farcm{\hbox{$.\mkern-4mu^\prime$}}
\def\farcs{\hbox{$.\!\!^{\prime\prime}$}} 


\begin{document}
%\initfloatingfigs


% -----
{\clearpage\thispagestyle{empty}}
\begin{center}
\ \\ {\LARGE Maria Messineo} \ \\
\ \\ {\LARGE Hidden knowledge in the Milky Way} \ \\
\vspace{1.5 cm}
born in Petralia Soprana (Italy)\\
in 1970 \\
\end{center}
% -----
%\frontmatter
\tableofcontents
%\listoffigures
%\listoftables
%\begin{thepreface}\index{Preface}
%\end{thepreface}
\mainmatter


   \chapter*{Hidden knowledge in the Milky Way }
\chaptermark{}
 
\begin{authorline}
        M.\ Messineo
        %\journal{Astronomy and Astrophysics (2004), submitted}
\end{authorline}
 
\begin{abstract} 
Here is proposed the  creation of a catalog 
of known Galactic red supergiants (RSGs)
based on Gaia DR3 data.  
The catalog of RSGs by \citet{messineo19}
will be revised and used as the training set 
for extracting new RSGs.
Furthermore, several $1^\circ \times 1^\circ$ 
fields across the Galactic plane 
($15^\circ$, $25^\circ$, $35^\circ$) will be chosen
and Gaia DR3 parameters 
(parallaxes, spectral energy distribution, spectral-types
temperatures, metallicity) of all 2MASS infrared bright stars
with colors of  late-type stars (AGBs and RSGs) 
will be analyzed, 
for extracting  clean samples of RSGs and for estimating
relative stellar counts and surface densities.
Eventually, but not less important, new infrared constrains
will be derived useful  for extracting   obscured
candidate RSGs.
\end{abstract}

\section{State of the art and preliminary work:\\
Bright late-type stars in the inner Galaxy}

DENIS and 2MASS catalogs  were released at the end of the 90's. 
That was very exciting, we had the first 
near-infrared image surveys of the entire Galactic plane.
Near-infrared data were combined with mid-infrared data from
ISOGAL, MSX, and GLIMPSE and the selection of individual Galactic 
stars based on their energy stellar distribution had started.
Amazing work has been carried out since then,
to discover that the Galactic disk is full of marvelous bubbles,
and that there is an old Bar at the center of the Galaxy.
We were searching for tracers, stellar tracers of the Galaxy,
to decompose it into basic components.
Asymptotic giant branch (AGB) stars and planetary nebulae were
the first stellar tracers of the central bar, the
stars of the central peanut box.

Nowadays, we have arrived at the epoch of archeoastronomy 
with millions of parallaxes, radial velocities, and spectra, 
becoming publicly available and  
ready to be used, as well as   time-series data.
The entire book of Astronomy needs
to be re-checked, everything has to be recomputed.
But still, something is missing, and
I am especially referring to the topic that
I have mostly followed, i.e., to
our knowledge of evolved late-type stars
and evolved massive stars in the most obscured
and central regions of the Milky Way.


Massive stars allow for a morphological and kinematic 
mapping of the youngest components of the Milky Way, 
such as spiral arms and the central 200 pc disk, and
it is commonly believed that the galactic Bar is 
devoid of it.  
Massive stars are a measure of the youth and 
thickness of the disk, and they can be used to map 
variations of chemical abundances in the disk. 
Among them, RSGs are cool stars 
with temperatures from 4500 K to 3000 K and 
ages from 8 to 30 Myr. 
In the inner Galaxy, uncertain distances and 
dust obscuration  hamper 
the detection of  RSGs, which remain hidden 
in a sea of AGB stars. AGBs have   properties similar
to those of RSGs, 
but lower initial masses and are much older 
(from  50 Myr to 12 Gyr). 
In order to study the  evolution and morphology 
of the Milky Way, {\it it is crucial to distinguish 
RSGs from AGB stars}, as  their spatial distributions
are predicted to be different in a barred gravitational
potential. Only a few thousand RSGs  
populate the Galaxy (less than 1,000 RSGs are known 
 and more than 5,000 are expected). 
But,  to enlarge this number a careful 
analysis of their   dissimilarities is required.
There is a large amount of Galactic data and  recent results  
to be reconsidered to improve our  tools.

Most of the forthcoming spectroscopic 
surveys (GALAH, 4MOST, LAMOST), and time series (e.g. LSST)
will be performed  with CCD  at optical wavelength, covering 
only to 0.7-0.9 \um. The new data should allow
for a precise classification of individual stars
and to build a 3D image of the portion of the Galaxy observed.
All the detailed knowledge acquired from the close 2-3 kpc,
must, then, be applied to the inner Galaxy, 
where most of the mass is.
This presented study  aims
to collect  new knowledge and to 
build an infrared transfer function.

\subsection{Evolved late-type stars}
In this writing, the term "late-type stars" is used
to indicate bright evolved stars cooler than 4600 K.
They divide in RSGs and AGBs. \\
RSGs are massive stars ($9 <$ masses $<40$ \Msun),  
i.e., stars without a degenerated state in their 
core, and are typically observed when burning He.\\
AGB stars are stars of low and intermediate
mass ( $<9$ \Msun) which are characterized by 
two layers of burning matter, He in the inner shell
and Hydrogen in the outer shell, while the central core 
of CO is in an electron-degenerate state.

AGB stars come with different chemistry.
Those with masses from 1 to 5 \Msun\ 
evolve from an O-rich envelope ($[$C/O$] <1$)
to a C-rich envelope ($[$C/O$] <1$), 
due to the dredge-up materials
that enrich the surface. S-stars are stars 
in an hybrid state, transiting toward a C-rich chemistry.



Late-type stars are often named  with terms indicating 
their variability type (e.g., Mira, semiregular, 
irregular, long period variables, 
and large amplitude variables)
or their envelope types (e.g., OH/IR stars).


\subsection{How are inner Galactic late-type stars
photometrically identified?}

A combination of near- and mid-infrared bands
allows us to identify mass-loosing late-type stars.
The availability of multi-wavelength photometry
and improved spatial resolutions 
(from 18.3 of MSX to a few arcs of GLIMPSE)
allows doing that quite straightforwardly nowadays.
Interstellar extinction reddens the stellar energy distribution.
Mass-loss increases with  redder stellar colours, from 
naked giants to enshrouded OH/IR stars which may have 
circumstellar extinction even higher than \Av=40 mag.
A good decomposition of the total interstellar extinction
must be performed in the two basic components 
(interstellar and  circumstellar).
In \citet{messineo02} and \citet{messineo05}, 
2MASS, DENIS, MSX, and ISOGAL
catalogs were used to successfully extract a sample of 
obscured SiO masing late-type stars and estimate their
intrinsic colours.

Nowadays, the availability of multi-wavelength data
allows for a better determination of the Galactic 
interstellar extinction laws, as well as for  good
templates of the stellar energy distribution.
One can define colors free of interstellar extinction which
directly indicate a certain type of intrinsic
stellar energy distribution (knowing the class of objects),
and even  distinguish  certain classes of objects. 
For example,  \citet{messineo12} analyzed near- 
and mid-infrared colors of evolved 
stars in the Galactic plane by using 2MASS and GLIMPSE data. 
We use the $Q1$ and $Q2$ parameters, which are colors
free of interstellar extinction. 
These parameters represent deviations 
of infrared color-color data points from the vector 
of interstellar extinction. 
It is possible to establish some criteria of selection 
for a particular inner Galactic class of objects; 
for example, Wolf Rayet (WR) stars and RSGs.  
40\% of the known RSGs 
appears bluer than Asymptotic Giant Branch 
stars (AGB) and may be identified  by their narrower 
range of $Q1$ and $Q2$.
The predictions were successfully tested
by \citet{messineo16} and \citet{messineo17}.
The color criteria were used to spectroscopically search 
in the GLIMPSE I North area of the inner Galaxy
with 2mass and GLIMPSE data.
About 50\% of the sample appeared to be consistent with 
RSGs. Distances, however, remain the biggest issue.
For obscured stars, usually distances are inferred 
using the interstellar extinction as a meter of distances.
Kinematic distances are not allowed in the central regions, 
being the Galaxy characterized by  non-circular motion.
However, these old photometric selections
based on  single epoch data appear now
 inadequate, because biased by a color cut and because based
on single epoch measurements.
Late-type stars vary often.
With the huge amount of new data and library recently developed,
and machine learning method, it is time to cross-correlate
all available information. 

\subsection{An infrared catalog of  known RSGs, mostly optically visible}
In order to build a large scale view and history of the Milky Way, 
I  spend a few years  making 
a compilation of know Galactic RSGs from the literature.
The catalog comprises about 1400 known candidate RSGs, 
which appeared in the literature at least once as class I stars
\citep{messineo19};
most of the entries were already included in the 
spectroscopic list by \citet{skiff14}. 
The infrared photometric catalog was crossmatched
with Gaia data  EDR3 \citep{messineo21z}, and good distances and 
luminosity  were inferred for 966 stars.
There is a  tail of 110 stars (11\%) brighter than the AGB 
luminosity limit (Mbol=$-7.1$ mag); 
at least  49\% of the stars are brighter than 
Mbol=$-5.0$ mag and  earlier than M4, which means
they are  more massive than 7 \Msun, 
making them likely RSGs. 
Unfortunately, 28\% of the sample appears to have luminosity below
the tip of the red giants.
By reordering it, a clean sample of RSGs
is obtained. This nearby training set is a
golden key for unlocking the inner Galaxy.
%%
%% Example of single figure
%%
\begin{figure}[ht!]
 \centering
 %%%\includegraphics[width=0.49\textwidth,clip]{calibsel_4.eps}   deltaG.eps   
  \includegraphics[width=0.8\textwidth,clip]{fig/mbolbin_dr3.eps}      
\caption{ \label{fig.lum-teff} 
This Figure is taken from \citet{messineo21z} ans is  an updated version 
of the figure number 7 appearing in \citet{messineo19}. The average
\Mbol\ values per spectral type are plotted versus the
\Teff\ values. The average \Mbol\ values estimated with distances 
from \citet{bailer21} are used. 
Black filled circles indicate the \Mbol\ values with EDR3 
data for the sample of reference RSGs 
(the \Mbol\ with DR2 values are shown in red). 
Cyan crosses show the average \Mbol\ values 
obtained with EDR3 data for class Ia and Iab stars 
(for comparison, the previously
\Mbol\ with DR2 values are shown in light orange).}
\end{figure}

\subsubsection{The luminosity-temperature relation}
In \citet{messineo19}, the average magnitudes of RSGs 
per spectral type are derived, and a revision with EDR3 Gaia data
is made by \citet{messineo21z}.
Interestingly, the stellar bolometric magnitudes 
appear to linearly correlate  with  the stellar temperatures, 
as shown in \ref{fig.lum-teff}.
This is a remarkable relation because it may allow 
astronomers to use RSGs as indicators of distances.


%%
%% Example of single figure
%%
\begin{figure}[ht!]
\begin{center} 
 %%%\includegraphics[width=0.49\textwidth,clip]{calibsel_4.eps}   deltaG.eps   
  \includegraphics[width=0.5\textwidth,clip]{fig/messineo_nxnn.fig2.eps}      
  \includegraphics[width=0.45\textwidth,clip]{fig/chathys-Iclass_plx.eps}      
\end{center}
\caption{ \label{fig.gaiaampl} 
{\bf Right panel:} As in \citet{messineo20var}, differences beween the maximum and minimum 
magnitudes measured in $G$-band versus spectral types. The legends list 
the average $\Delta G$ measured in the areas of the HR-diagram defined 
by \citet{messineo19} (from A to F). {\bf Left panel:}
\Mk\ values vs. periods of Galactic RSGs. 
Periods are from \citet{chatys19} and are marked with black diamonds. 
Gaia variable late-type from the catalog of \citet{messineo19}
are marked with filled circles in red.  
The distances are from Gaia EDR3 \citep{messineo21z}.
}
\end{figure}


\subsubsection{The period-luminosity relation of RSGs}\label{law}

A well-studied sample of variable RSGs is listed in the work of 
\citet{chatys19} and \citet{kiss06}, with data covering 
61 years of observations.
The authors analyzed data from the AAVSO archive to determine
the stellar periods. RSGs may have a short
period (100-1000 d) as well as a long period
($>2000$ d), as already seen in bright Mira AGBs.  
The periods of Galactic RSGs pulsating in 
their fundamental mode are distributed around the relations
derived by  \citet{soreisam18}.
The RSG period-luminosity of stars pulsating in the 
fundamental mode exists and appears to be independent
of metallicity.  In the Milky Way, it is particularly difficult to
determine it because of the small amplitudes, extinction, and 
uncertain distances. Furthermore, RSGs have multiple periods,
and often the  algorithm used to analyze the periodicity 
may only yield the highest power frequency, which does not 
necessarily correspond to the fundamental mode. 

The fundamental mode pulsators in \citet{chatys19} 
appear to  fall onto the  period-luminosity 
relation found in M31 by \citet{soreisam18}, 
but the temporal baseline is 60 years. 
%The AAVSO periods are calculated with data taken over 61 years,
%and the absolute magnitudes are calculated with Gaia EDR3
%parallaxes \citet{messineo21z}. 
Baselines  of a few years, such as those delivered by Gaia,
will allows to determine only the short periods, and
may have a period uncertainty of more than 100 d
as Fig \ref{fig.gaiaampl} suggests.
%{\it The long span of time 
%is the  essential tool
%for using the period-luminosity relation 
%as a tool to infer distances.}

While the periods of RSGs range from 100 to 1000, similarly to
Mira AGB stars, the amplitudes of Galactic RSGs are
smaller ($<0.8$ in $G$-band) than those of Mira AGBs (up to 8 mag).
Therefore, {\it 
amplitudes are a very useful tool  to separate RSGs from AGBs}; 
furthermore,  amplitudes can be easily estimated. 
About 10\% of the stars in  Messineo's catalog  
are flagged as long-period variables \citep{holl18}. 
Those variables coincide  with later spectral types from K5 to M7
\citep{messineo20var};
their average variation in $G$-band is 0.51 mag with a 
dispersion around the mean of 0.38 mag,
as shown in Fig. \ref{fig.gaiaampl}. 


\subsection{Spectroscopic identification of RSGs}

A large amount of optical spectroscopic surveys (Galah, 
LAMOST, 4MOST) at high and medium resolution 
is soon becoming available. This will constitute 
{\bf a great test-bed for late-type stars},  
to extract the useful differentiating properties 
for the inner Galaxy where most of the Galactic 
mass is located.\\
The  GALAH survey is made with the HERMES detector
on the Anglo-Australian Telescope and will release more 
than 1 million spectra of Galactic southern stars 
at R=28,000 in the 4700 to 7600 nm \citep[e.g.][]{sharma20}. 
The 4MOST detector, installed on the ESO/Vista telescope,
will survey the sky to acquire more than 20 million spectra 
at R≈5000 (from 390 to 1000 nm) and R≈20,000 
(from 395 to 456.5 nm \& from 587 to 673 nm) 
of southern stars, starting on 2024 
\citep[e.g.,][]{deJong12}.
Gaia, LAMOST, and 4MOST are particularly useful because,
by covering till 1 \um, they provide a bridge between the
study of the nearby regions and the central Galaxy.


Since late-type stars are cold,  
they are naturally brighter at infrared light, where 
extinction  is about 10 times lower than 
at optical wavelengths.  
Infrared detectors allow us to  penetrate 
the inner Galactic regions.
However, the spectra of RSGs and AGBs may appear very similar,
and large spectral coverage and high-resolution are required to
distinguish them. 
Usually, the strength of CO band-heads at 2.29 \um\ and  
the shape of the continuum, which 
changes due to  water vapor absorption, are used
to detect RSGs. However, 
in about 7\% of RSGs water vapor is also seen.
Furthermore, single epoch $K$-band observations 
alone  do not allow to distinguish RSGs from all subtypes of AGB stars.
While Mira AGB variables are easily identified by their strong water
absorption, S-type stars   may look very similar  to RSGs. 
Little is also known about the spectral appearance of 
super AGB stars (s-AGB) stars.
AGBs  are far more numerous
than  RSGs. An in-depth spectroscopic study of these bright 
luminous late-type stars is essential 
for the proper construction of the Galactic
luminosity functions of AGBs and RSGs, expecially 
for the central regions,
where parallaxes are not available.\\
Recently, \citet{messineo21} analyzed low-resolution spectra 
(R=2000) of late-type stars covering from 0.7 to 
2.4 \um, 
and identified five lines correlating in strengths 
with the stellar luminosity (from 1.16 μm to 1.29 μm 
due to  Fe, Ti, Mn, and Ca). This is very promising  
as these atomic lines help us 
to distinguish RSGs from AGBs in the inner Galaxy. 
 







\subsection{Project-related publications}

\begin{enumerate}
\item    {\bf Messineo M.}, Habing H., Sjouwerman L., Omont A. \&
    Menten K., {\it 86 GHz SiO maser survey of late-type stars in the Inner Galaxy. I.
    Observational data}, 2002, A\&A 393, 115.
    
\item    {\bf Messineo M.}, Habing H.,   Menten K., Omont A. \& Sjouwerman L. {\it
    86 GHz SiO maser survey of late-type stars in the Inner Galaxy. II.
    Infrared photometry of the SiO target stars}, 2004, A\&A 418, 103.
    
\item    {\bf Messineo M.}, PhD thesis, {\it Late-type Giants in the Inner Galaxy}, 
2004, Leiden University.


\item    {\bf Messineo M.}, Habing H.,   Menten K., Omont A. , Sjouwerman L. \& Bertoldi, F.{\it
    86 GHz SiO maser survey of late-type stars in the Inner Galaxy. III.
    Interstellar extinction and colours of the SiO targets}, 2005, A\&A 435, 575.



\item {\bf Messineo Maria}; Menten, K.,  Churchwell, E., and  Habing, H.,
{\it  Near- and Mid-Infrared colors of  evolved stars in the Galactic plane.
The $Q1$ and $Q2$ parameters.}, 2012 A\&A, 537,10

\item Verheyen L., {\bf Messineo M.}, Menten K.M.,
{\it SiO maser emission from red supergiants across the Galaxy. I. Targets in massive
star clusters.}, 2012, A\&A 541,36


		
\item {\bf Maria Messineo}, Qingfeng Zhu, Valentin D. Ivanov, Karl~M. Menten, Ben Davies, 
Donald F.  Figer, Rolf P. Kudritzki, and  C.-H. Rosie Chen,
{\it Near-infrared Spectroscopy of a few candidate Red Supergiant Stars in  Clusters}, 2014, A\&A, 571, 43.


\item
{\bf Messineo M.}, Zhu Q., Menten K.M., Ivanov V.D., Figer D.F., Kudritzki R.-P. \& Chen C.-H. R.,
{\it Discovery of an Extraordinary Number of Red Supergiants in the Inner Galaxy,} 2016, ApJL, 822, 5.

\item
{\bf Messineo M.}, Zhu Q., Menten K.M., Ivanov V.D., Figer D.F., Kudritzki R.-P. \& Chen C.-H. R.,
{\it Red Supergiants in the Inner Galaxy: Stellar Properties,} 2017,  ApJ, 836, 65.

\item    {\bf Messineo M.}, Habing H. J., Sjouwerman L. O., Omont A. \&
    Menten K. M., {\it 86 GHz SiO maser survey of late-type stars in the Inner Galaxy. IV.
    SiO emission and infrared data for sources 
    in the Scutum and Sagittarius-Carina arms, $20^\circ < l < 50^\circ$.}, 2018, A\&A,619,35.

\item {\bf Messineo M.}, Brown, A. G. A.
{\it A Catalog of Known Galactic K-M Stars of Class I Candidate Red Supergiants in Gaia DR2.},
2019, AJ, 158, 20.

\item {\bf Messineo M.}, Sjouwerman, L. O., Habing, H. J., Omont A.
{\it 86-GHz SiO masers in Galactic centre OH/IR stars.}, 
2020, PASJ, 72, 63.



\item  {\bf Messineo M.}, Figer D.F., Kudritzki R.-P., Zhu Q., Menten K. M.,
Ivanov V.D., \& Chen C.-H. R.
{\it New infrared spectral indices of luminous cold stars: 
from early early K to M-type.} 2021, AJ, 162, 187.

\end{enumerate}

\begin{enumerate}

\item  {\bf Messineo M.}, Petr-Gotzens M.G., Schuller F., Menten K.M., Habing H.J., Kissler-Patig M., Modigliani A., Reunanen, J.
{\it Integral-field spectroscopy of the Galactic cluster [DBS2003]8. Discovery of an ultra-compact HII region and its ionizing star in 
the bright rimmed cloud SFO49},  2007, A\&A, 472, 471.


\item  {\bf Messineo M.}, Figer D.F., Davies B., Rich R.M.,  Valenti  E., Kudritzki, R.P.
{\it Discovery of a Young Massive Stellar Cluster near HESS J1813-178}, 2008, ApJ, 683, 155.

\item  {\bf Messineo M.}, Davies B., Ivanov V.D., Figer D.F., Schuller F., Habing H.J.,
Menten K.M., Petr-Gotzens M.G. 
{\it Near-Infrared Spectra of Galactic Stellar Clusters Detected on Spitzer/GLIMPSE Images.}, 2009, ApJ, 697, 701


\item {\bf Messineo, Maria}; Figer, Donald F.; Davies, Ben; Kudritzki, R. P.; Rich, R. Michael; MacKenty, John; 
Trombley, Christine, {\it HST/NICMOS observations of the GLIMPSE9 stellar cluster}, 2010, ApJ, 708, 1241

\item {\bf Messineo, Maria}; Davies, Ben; Figer, Donald F.; Kudritzki, R. P.; Valenti, Elena; Trombley, Christine; Najarro, F.; Rich, R. Michael,
{\it Massive Stars in the Cl 1813-178 Cluster: An Episode Of Massive Star Formation 
in the W33 Complex",}  2011 ApJ, 733, 41.

     
\item {\bf M. Messineo}, K.~M. Menten, D.~F. Figer, B. Davies, J.~S. Clark, V.~D. Ivanov,
R.-P. Kudritzki, R.M. Rich, J.W. MacKenty, C. Trombley
{\it  Massive stars in the giant molecular cloud G23.3-0.3 and W41.}, 2014,  A\&A, 569, 20.


\item {\bf Maria Messineo}, Donald F. Figer, J. Simon Clark, Rolf-P. Kudritzki, F. Najarro, R. Michael Rich,
Karl M. Menten, Valentin D. Ivanov, Elena Valenti, Christine Trombley, C.-H. Rosie Chen, and Ben Davies,
{\it Massive stars in the W33 giant molecular complex}, 2015, ApJ, 805, 110.
    
\item  {\bf Messineo M.}, Menten K. M., Figer D.F., Chen C.-H. R., \& Rich R.M.
    {\it Detections of the massive stars in the cluster MCM2005B77, in the star-forming
    regions  GRS G331.34−00.36 (S62) AND GRS G337.92−00.48 (S36).}, 2018, ApJ 862,10.
     


\item {\bf Messineo M.}, Menten K.M., Figer D.F., Clark J.S. 
{\it Massive Stars in Molecular Clouds Rich in High-energy Sources: 
The Bridge of G332.809-0.132 and CS 78 in NGC 6334.}, 
2020, AJ, 160, 65.
\end{enumerate}




\section{  Research plan, including proposed research methods}
\subsection{ Anticipated total duration of the project}
Three years.

\subsection{The Ultimate Objective}
The work aims to optimize our knowledge on 
classification of Galactic late-type stars.
 With a comparative method, 
the optical properties and infrared 
of RSGs and AGBs stars
will be revised. 
The  new findings at optical wavelengths  should be translated
at infrared wavelengths to define a new strategy 
to distinguish RSGs from  AGBs
and to build a catalog of candidate RSGs in the inner Galaxy.
Such a catalog may serve as  the efficient input for 
follow-up spectroscopic surveys.
Such a catalog, alone, may provide a direct evidence for a
different distribution on the Galactic plane
of the two categories of stars.
{\it It is a key project for the still hidden formation history
of the Milky Way. Where are the central Disk 
high-velocity RSGs? is there
a population of RSGs populating the Galactic bar?}
The distribution of RSGs may be a step function of age,
and the low-end mass, 8-9 \Msun, could be populating the Bar.
Stars of different ages are distributed differently on the
Galactic plane.
It may be misleading to talk about the initial 
mass function of RSGs, without being able to clearly 
distinguish RSGs from AGBs. 

%Furthermore, one should talk perhaps
%only about global mass function, due to the small number
%of expected RSGs, statistical fluctuations, and 
%age and metallicity of each Galactic components.


\subsection{ Work program including proposed research methods}
The money here request should serve to 
finance three years of work.
\begin{itemize}
\item
The first step (during the first and the second year) is to review the 
properties of nearby RSGs.
The catalog of 1400 known bright luminous late-type stars of 
\citet{messineo19} will be updated with
distances and parameters 
(Temperatures, gravity, metallicity, periods, amplitudes) 
released from the DR3 Gaia data. 
As in \citet{messineo19}, this will be a fast compilation
soon available to everybody for more detailed analysis.\\
The parameters of the 470 million stars with Gaia DR3 BP/RP spectra
can be exploited to extract new RSGs.\\
\item
The second step (during the second and the third year) 
is to define reference stars in fields
a moderate interstellar extinction
which are easily accessible at both optical and 
infrared wavelengths. 
Reference stars (AGBs and RSGs) with known parameters 
will be the training set.  
Distances, extinction, luminosity, bolometric corrections,
and mass-loss will be estimated.
Improved infrared diagnostics to identify new RSGs will be established.
Surface density per stellar type will be estimated.\\
\item
%The application of the new relations/criteria 
%should automatically
%generate the catalog of obscured candidate RSGs in the
%inner Galactic  
%with estimated parameters 
%(e.g. extinction, distances, mass-loss, bolometric corrections). 
%The search will be limited to point sources with
%mid-infrared detections in the WISE and Spitzer catalogs
%(second and third year).\\

New candidate RSGs (without parallaxes) 
will be also extracted and made available to everyone 
for high-resolution infrared spectroscopy.
Infrared spectroscopy is still quite expansive, 
and a detection rate of 80\% is  preferable 
for targeted surveys for  good use of telescope time. 
A multi-slit spectrograph
still requires a pre-selection of targets.
Blind surveys are interesting, but they may only carried out 
in  small selected areas of the sky.
\end{itemize}
 
\subsubsection{First step: Revised catalog of RSGs based on DR3}\label{law}

A new version of the Gaia data release, DR3, will be released 
on the upcoming June, 
13th, 2022\footnote{https://www.cosmos.esa.int/web/gaia/dr3}.
Besides the updated parallax values, the new release
will provide new products. Astrometry parameters and 
source classification for 1.59 billion sources will be 
released, radial velocities from 33 million, as well as 
astrophysical parameters 
(spectral types, Teff, logg, [M/H], extinction) 
for 470 million objects based on the low-resolution BP/RP spectra. 
An updated catalog of variables will be also released, 
from 86,000 LPV in DR2 to 1,700,000 in DR3.
The first catalog of binary parameters with 813,687
objects will be released.
Light curves for 1,720,588 late-type stars.

%\subsubsection{Revised Absolute Magnitude per spectral-types}\label{law}
From a compilation of 1400 stars listed in the literature 
as a possible RSGs, \citet{messineo19}  found
889 sources with good Gaia DR2 parallaxes, 
while \citet{messineo21z}
retrieved 966  good Gaia EDR3 parallaxes. 
The upcoming DR3 release (1.5 billion sources) may have 
parallaxes for most of the 1400 candidates in Messineo's catalog.
Furthermore, the Gaia DR3 by delivering temperatures,  
spectral types, and gravity of 470 million stars with
low-resolution BP/RP spectra will provide
the first Gaia selected catalog of RSGs.
Stars with $[M/H] <0.5$ dex and 
logg $< 1$ cm s$^{-2}$, cool and with $G$-band amplitudes
$< 0.5$ mag should be regarded
as RSGs. This test sample may already allow us  
to prepare for the first Gaia initial mass 
function of Galactic RSGs. 

The sample of reference stars Galactic RSGs  studied in
\citet{messineo19} allowed us to obtain, for the first,   
estimates of the average absolute magnitudes of RSGs 
per bin of 1 spectral type and has revealed the
existence of a clear relation between stellar temperatures
and stellar luminosities, making them suitable direct
indicators of distances for extragalactic studies.
The DR3 release will  allow to refine
the average magnitudes per spectral type 
and the luminosity-temperature relation. 

A revised catalog with a summary  of properties of known RSGs 
(e.g. average magnitudes per spectral type,
luminosity-temperature, luminosity-period,  
and luminosity-gravity relations)
will be delivered within the first year
of the fellowship with a publication.  

\subsubsection{Second step: The two extreme ends and contaminants}
Have we learned to go down in luminosity and 
with confidence separate AGBs from 8\Msun\ RSGs?
and where are they? being old, they could be distributed
differently than massive RSGs; for example, they could 
be  populating the long Bar as AGB stars do.
Are the low-mass end RSGs tracing the near-end side 
of the Bar?
This important issue requires  careful analysis,
based on  secure stellar classes and spectral types.

The second year of the fellowship should be
spent comparing the properties of bright late-type 
stars, i.e., super AGB stars (s-AGB), S-stars, Mira-AGBs, 
and RSGs in a few designed fields of the Galaxy.
Gaia spectral types and    Gaia BP/RP spectra 
will allow a precise study of the stellar energy distribution.
A large amount of broad-band photometry is already available
from optical to infrared (e.g. SDSS, PANSTARRS, 2MASS, WISE).
High-resolution spectra from LAMOST, 4MOST, and APOGEE could 
supplement the Gaia low-resolution spectra 
and provide spectral diagnostic
in the red part of the spectrum.
Existing catalogs of late-type stars
will be  used to define/check the  set of reference stars,
for example the LAMOST S-stars catalog by \citet{chen22},
the compilation of Galactic AGBs  by \citet{suh21}.
Luminosities will be estimated, and periods, amplitudes, and 
spectroscopic parameters will be used 
to estimate the luminosity class. 
Relative counts of bright late-type stars 
(brighter than the tip of the red giant branch)
may be estimated in each studied field.

Fields $1\degr \times 1\degr $ in size will be selected
along the Galactic plane 
(e.g., 45\degr, 35\degr, 25\degr, and 15\degr)
at moderate extinction (\Ak=0.8-1.2 mag), where 
the surface density of each type can be estimated. 
In each field,  clean samples of RSGs, O-rich AGB, C-rich AGB, 
and S-stars, will be extracted and properties and 
relative density analysed. All properties will 
be analyzed in a comparative manner (counts, luminosity ranges,
intrinsic colours, amplitudes, periods). 

\subsubsection{New catalogs of obscured candidate RSGs}
From the detailed analysis in the selected fields,
with the resulting tables in hands and count ratios, 
it may be possible to obtain improved and more realistic
infrared diagnostics/cuts for RSG selection in 
the obscured inner Galaxy
($-10^\circ <$ long $<+10^\circ$), where parallaxes
are missing; for example, by building guesses on distances and 
intrinsic stellar energy distribution, 
and using any existing information
on amplitudes. In summary, candidate RSGs can be extracted and the 
fraction of contaminants will
have an empirical-made function.


\subsubsection{Strange objects: Thorne-\.{Z}ytkow Objects}

The period-luminosity diagram is a good diagnostic tool
to locate Thorne-\.{Z}ytkow Objects \citep[e.g.][]{grady20,demarchi21}. 
Those rare objects
are the merger product of a binary system with a neutron star. 
Their appearance resemble RSGs, but there are peculiar 
chemical signatures in their spectra. 
There is only one such  object in the Magellanic Clouds 
\citep{levesque14},
and it has a large amplitude ($>4$ mag), which distinguishes 
it from other RSGs. Bright objects lying in the RSGs sequence
with large amplitudes must be considered for high-resolution
spectroscopy. In  recent years, 
these strange objects have earned more attention
due to the new possibility to measure gravitational waves.
It is possible that the LSST survey and Gaia survey 
may unveil more of them,
as about 100-200 such objects
are expected to populate the  Milky Way \citep[e.g.][]{demarchi21}.

%In the DR2 Gaia release there are only light-curves
%for 137 out of 1406 (10\%) late-type stars 
%listed by \citet{messineo19}.

%The new Gaia releases with likely release periods for 50\%
%of the sample of RSGs collected by \citet{messineo19}. 
%Furthermore, new RSGs will be identified by Gaia.
%It is possible that a new Galactic Thorne-Żytkow Object
%will be identified, 

%$\subsubsection{Concluding remarks: New learned tools to 
%$select obscured RSGs}


%$\subsubsection{A representative application work: the study 
%$of one  Galactic association with identified RSGs 
%$and early massive stars. Training the tools.}




\section{Handling of research data}
Tables and catalogues will be described in referred
publications, and therefore stored in the Journal archive.
Furthermore, the astronomical data products  will also
be made available via repositories on  github 
(github gives a warranty of 20 years on free repositories)
and/or on the astronomical Vizier database
$http://vizier.u-strasbg.fr/$.

\section{Relevance of sex, gender and/or diversity}

Astronomers do not discriminate in gender, and race.  
Despite the great technical skills,  we do not let the new 
era hybridize us, keeping high our human spirit.

At the age of 51 years, Maria  returned back to 
Europe. 
%She was very welcome at the Chinese Institution,
%but, she was  not within the age limits for most 
%of the jobs there offered and her research contract
%was not renewable. 
Currently, She is applying for 
several astronomical positions, as well as for 
financial supports for her research. 
COVID  is making the 
re-entering process quite slow, and  now the 
Ukrainian war is (at least temporarily) splitting
the unified world of scientists.


\section{Bibliography}  		     
\bibliographystyle{aa}
\bibliography{biblio}

\section{Relevance of project to research career objectives}

Despite being a simple program, devoted to create an
updated catalog of RSGs with collection and 
comparison of existing data, the program will
contribute to a fundamental understanding
of the history of our Galaxy evolution.
The main aim is to improve the photometric identification
of RSGs and AGBs, using  new available
information, for example  on variability and extinction.

It is believed that the Bar is old and dissolving 
only populated by giants and AGB stars.
Stellar migrations may account for a small 
population of bright RSGs populating the Galactic Bar.
Also one need to recall that what today 
we call RSG population is only the bright 
tail of such a population (12-25 \Msun), as we are 
still unable to locate RSGs of 8-9 \Msun\
which are supposedly much more numerous.


\section{ Reasons for selecting host institution(s)}
The Leibniz Institute for Astrophysics Potsdam (AIP)
has a long tradition of stellar research and
Milky Way stellar populations. The experts
currently hired  make the science environment
optimal for carrying out the research here prospected.

\section{ Desired state date of fellowship}
The reviewers may need a few months to decide.
The work should be started as soon as possible,
preferably before the beginning of the 
next academic year (2022-2023).

\section{ Additional publication costs}
So far, Maria has published in  Peer Reviewed Journals,
A\&A, AJ, ApJ, PASJ.
Only A\&A is free of charge
for astronomers affiliated to a European Institution.
Usually, I choose the Journal depending on the topic
and previous publications. 
It would be nice to have a badget
of 3000 euro (about 2 papers) to warranty
publication of any produced material. 

\section{ Additional funding}
There are no other grant money that I can use.

\section{ Regulation} 
Maria is  planning to consider other funding
for the same research only after having received your answer.
She  is aware of your regulation:
" in the case of multiple awards, accepting such a fellowship 
before or after approval of the DFG fellowship precludes your 
being able to accept fellowship funding from the DFG".

\section{ Career plans}
Maria would like to remain a researcher for the entire working life.
She has at least 15 more years of work head, and she must 
improve her skills to find funding grants. 

\newpage
\section{  Curriculum Vitae }
\vspace{1. cm}
\noindent {\bf Maria Messineo}

\vspace{1. cm}
Maria Messineo has extensively worked on Galactic stellar populations,
with major focus on Galactic structure and evolution.
She started her science activity in  1996, when she builded a 
database of existing photometric data for 61 Globular Clusters 
to determine  new metallicity indicators. 
During the PhD she analyzed a sample of color-selected AGB stars, 
and carried a search for SiO masers.  
The maser kinematics revealed for the first time a clear 
stellar component belonging to the central 200-pc nuclear disk 
of the Milky Way.   From 2004 Maria is actively  
studying   massive stars in the Galactic plane, 
mainly at infrared wavelenghts. In particular, she
has experimented several methods of  photometric source
selection, and reported a successful survey of inner 
Galactic RSGs. She is currently learning
about time series to implements more complex, but efficient
selections. Due to COVID and war, she has 
self-sponsored herself for this last semenster. 


\vspace{1.2 cm}
\noindent Education\\
PhD in Astronomy 2004, Leiden Observatory, Netherlands\\
Master in Astronomy 1997, University of  Bologna, Italy\\

\vspace{1.2 cm}
\noindent Appointments\\
11/2021 - present : USTC affiliated, and Freelancer in Brandenburg \\
10/2015 - 11/2021 : Researcher at USTC\\
10/2010 - 10/2015 : Researcher at the Max Planck Institut fuer Radioastronomy\\
10/2009 - 09/2010 : ESA fellow \\
07/2007 - 07/2009 : Postdoctoral researcher at the Rochester Institute of Technology\\
09/2004 - 07/2007 : ESO fellow\\
12/1999 - 06/2004 : PhD at the University of Leiden\\
04/1999 - 11/1999 : Research Assistant at the Astronomical Observatory of Bologna\\
06/1998 - 11/1998 : Summer Research Assistant at the Space Telescope Science Institute \\
         			     
				
\printindex
\end{document}
